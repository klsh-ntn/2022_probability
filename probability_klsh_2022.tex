% arara: xelatex: {shell: yes}
% %arara: biber
% %arara: xelatex: {shell: yes}
% %arara: xelatex: {shell: yes}

\documentclass[12pt]{article}

\usepackage{hyperref} % гиперссылки

\usepackage{tikz} % картинки в tikz
\usetikzlibrary{arrows.meta} % tikz-прибамбас для рисовки стрелочек подлиннее

\usepackage{microtype} % свешивание пунктуации

\usepackage{array} % для столбцов фиксированной ширины

\usepackage{indentfirst} % отступ в первом параграфе

\usepackage{sectsty} % для центрирования названий частей
\allsectionsfont{\centering}

\usepackage{amsmath} % куча стандартных математических плюшек
\usepackage{amssymb} % символы
\usepackage{amsthm} % теоремки

\usepackage{comment} % добавление длинных комментариев

\usepackage[top=2cm, left=1.2cm, right=1.2cm, bottom=2cm]{geometry} % размер текста на странице

\usepackage{lastpage} % чтобы узнать номер последней страницы

\usepackage{enumitem} % дополнительные плюшки для списков
%  например \begin{enumerate}[resume] позволяет продолжить нумерацию в новом списке

\usepackage{caption} % что-то делает с подписями рисунков :)

\usepackage{qcircuit} % для рисовки квантовых диаграмм
\usepackage{physics} % бракеты

\usepackage{answers} % разделение условий и ответов в упражнениях


\usepackage{fancyhdr} % весёлые колонтитулы
\pagestyle{fancy}
\lhead{Теория невероятностей}
\chead{}
\rhead{КЛШ-2022}
\lfoot{}
\cfoot{}
\rfoot{\thepage/\pageref{LastPage}}
\renewcommand{\headrulewidth}{0.4pt}
\renewcommand{\footrulewidth}{0.4pt}



\usepackage{todonotes} % для вставки в документ заметок о том, что осталось сделать
% \todo{Здесь надо коэффициенты исправить}
% \missingfigure{Здесь будет Последний день Помпеи}
% \listoftodos — печатает все поставленные \todo'шки



\usepackage{booktabs} % красивые таблицы
% заповеди из докупентации:
% 1. Не используйте вертикальные линни
% 2. Не используйте двойные линии
% 3. Единицы измерения - в шапку таблицы
% 4. Не сокращайте .1 вместо 0.1
% 5. Повторяющееся значение повторяйте, а не говорите "то же"



\usepackage{fontspec} % что-то про шрифты?
\usepackage{polyglossia} % русификация xelatex

\setmainlanguage{russian}
\setotherlanguages{english}

% download "Linux Libertine" fonts:
% http://www.linuxlibertine.org/index.php?id=91&L=1
\setmainfont{Linux Libertine O} % or Helvetica, Arial, Cambria
% why do we need \newfontfamily:
% http://tex.stackexchange.com/questions/91507/
\newfontfamily{\cyrillicfonttt}{Linux Libertine O}

\AddEnumerateCounter{\asbuk}{\russian@alph}{щ} % для списков с русскими буквами
\setlist[enumerate, 2]{label=\asbuk*),ref=\asbuk*}

%% эконометрические сокращения
\DeclareMathOperator{\Cov}{Cov}
\DeclareMathOperator{\Arg}{Arg}
\DeclareMathOperator{\Corr}{Corr}
\DeclareMathOperator{\Var}{Var}
\DeclareMathOperator{\E}{\mathbb{E}}
\newcommand \hVar{\widehat{\Var}}
\newcommand \hCorr{\widehat{\Corr}}
\newcommand \hCov{\widehat{\Cov}}
\newcommand \cN{\mathcal{N}}
\let\P\relax
\DeclareMathOperator{\P}{\mathbb{P}}

\usepackage{multicol}

\usepackage[bibencoding = auto,
backend = biber,
sorting = none,
style=alphabetic]{biblatex}

\addbibresource{forecast_everything.bib}



% делаем короче интервал в списках
\setlength{\itemsep}{0pt}
\setlength{\parskip}{0pt}
\setlength{\parsep}{0pt}




\Newassociation{sol}{solution}{solution_file}
% sol --- имя окружения внутри задач
% solution --- имя окружения внутри solution_file
% solution_file --- имя файла в который будет идти запись решений
% можно изменить далее по ходу
\Opensolutionfile{solution_file}[all_solutions]
% в квадратных скобках фактическое имя файла

% магия для автоматических гиперссылок задача-решение
\newlist{myenum}{enumerate}{3}
% \newcounter{problem}[chapter] % нумерация задач внутри глав
\newcounter{problem}[section]

\newenvironment{problem}%
{%
\refstepcounter{problem}%
%  hyperlink to solution
     \hypertarget{problem:{\thesection.\theproblem}}{} % нумерация внутри глав
     % \hypertarget{problem:{\theproblem}}{}
     \Writetofile{solution_file}{\protect\hypertarget{soln:\thesection.\theproblem}{}}
     %\Writetofile{solution_file}{\protect\hypertarget{soln:\theproblem}{}}
     \begin{myenum}[label=\bfseries\protect\hyperlink{soln:\thesection.\theproblem}{\thesection.\theproblem},ref=\thesection.\theproblem]
     % \begin{myenum}[label=\bfseries\protect\hyperlink{soln:\theproblem}{\theproblem},ref=\theproblem]
     \item%
    }%
    {%
    \end{myenum}}
% для гиперссылок обратно надо переопределять окружение
% это происходит непосредственно перед подключением файла с решениями



\theoremstyle{definition}
\newtheorem{definition}{Определение}



\begin{document}

\tableofcontents{}

\section*{Анонс}
...

\newpage
\setcounter{section}{0}

\section{Джентельменское соглашение}

Реальность и модель. 

Множество $\Omega$ — список всех исходов. 

Не знаешь как решать — рисуй дерево!

Пример дерева. Красная шапочка, все дороги ведут в усадьбу Бабушке, а встреченный волк дарит цветы, всего пять дорог.

Событие, случайная величина. 

Интуитивно: событие — утверждение об исходе эксперимента, которое может быть истинным или ложным, случайная величина — числовое описание исхода эксперимента.

Перемножение вероятностей. А почему они умножаются?

Случайные величины: время в пути, число левых поворотов, число встреченных волков. 

Табличка распределения. 

Вероятность, математическое ожидание. 

Формально, $A \subseteq \Omega$, $X : \Omega \to \mathbb R$.

Упражнение. Подбрасываем монетку до двух орлов подряд, $X$ — число бросков. 
Найдите $\P(X=2)$, $\P(X=3)$, $\P(X=4)$, $\P(X=5)$, $\P(X>5)$ и $a=\E(X)$.

Обращаем внимание: ожидаемое число левых поворотов и наиболее вероятное число левых поворотов — это разные величины.

При поиске вероятностей использовали степени букв! Это хорошая идея, чтобы потом писать производящие функции!
Например,
\[
\P(THHHHTT) = \P(TH^4T^2)  
\]

Угадайте $a$. 

Решаем через одну неизвестную:
\[
a = 0.5(a + 1) + 0.25(a + 2) + 0.25 \cdot 2  
\]

О школьниках: на первом занятии было 17 человек. 


\section{Подборка задач}

Подборка задач для распечатки, умещается на один лист, разумно распечатать компактно, 
упаковав две А5 страницы на один лист А4 и потом разрезать.

% \newpage % для выделения отдельной страницы
\begin{enumerate}
  \item  Неправильную монетку с вероятностью «орла» равной $0.7$ подбрасывают до первого «орла».
  Чему равно среднее количество подбрасываний?  Орлов? Решек? Какова вероятность чётного числа бросков? 

  \item Подбрасываем монетку бесконечное количество раз. 
  \begin{enumerate}
    \item Сколько в среднем ждать до появления последовательности HTT? А до THT? 
    \item Какова вероятность того, что последовательность HTT будет выкинута раньше THT?
    \item Сколько в среднем ждать до появления HTT или THT?
  \end{enumerate}
    

  \item Роберт Адлер нажимает на кнопку «Вкл/Выкл» на пульте дистанционного
управления телевизором. Изначально телевизор включён. Батарейки
у пульта садятся, поэтому в первый раз кнопка срабатывает с вероятностью
1/2, а далее вероятность срабатывания кнопки падает, причем падает совершенное непредсказуемым образом. 

\begin{enumerate}
  \item Какова вероятность того, что после 2022 нажатий телевизор окажется включён?
  \item Кто такой Роберт Адлер?
\end{enumerate}

\item Какова вероятность того, что у здесь собравшихся есть хотя бы одно совпадение по дням рождения?
% На турнире команд ФМТ одновременно в основном составе участвует 88 школьников. 
% Какова вероятность того, что у них есть хотя бы одно совпадение по дням рождения?
А если бы нас собралось 50 человек?

\item Илья Муромец стоит на развилке у камня. От камня начинаются
ещё три дороги. Каждая из дорог оканчивается камнем.
И от каждого камня начинаются ещё три дороги. И каждые те три
дороги оканчиваются камнем\ldots И так далее до бесконечности. На
каждой дороге живёт трёхголовый Змей Горыныч. Каждый Змей
Горыныч бодрствует независимо от других с вероятностью одна третья. 

\begin{enumerate}
  \item Какова вероятность того, что ИМ встретит ЗГ, если выбирает дороги равновероятно?
  \item Какова вероятность того, что у ИМ \textit{существует} хотя бы один путь, избегающий встречи с бодрствующими ЗГ?
\end{enumerate}

\item В каждой вершине треугольника по ёжику. Каждую минуту с вероятностью $0.5$ каждый ежик
независимо от других двигается по часовой стрелке, с вероятностью
$0.5$ — против часовой стрелки.
Обозначим $T$ — время до встречи всех ежей в одной вершине.

Найдите $\P(T=1)$, $\P(T=2)$, $\P(T=3)$, $\E(T)$.

\item Саша и Маша поженились и решили, что будут заводить новых детей до тех пор,
пока в их семье не будут дети обоих полов. Обозначим $X$ — количество детей в их семье.
Найдите $\P(X=4)$, $\E(X)$.

\item Вася подкидывает кубик\index{кубик} до тех пор, пока на кубике не выпадет единица, или пока он сам не скажет «Стоп». 
Вася получает столько рублей, сколько выпало на кубике при последнем броске. 
Вася хочет максимизировать свой ожидаемый выигрыш.
\begin{enumerate}
\item Как выглядит оптимальная стратегия? Чему равен ожидаемый выигрыш при использовании оптимальной стратегии?
\item Какова средняя продолжительность игры при использовании оптимальной стратегии?
\item Как выглядит оптимальная стратегия и чему равен ожидаемый выигрыш, если за каждое
подбрасывание Вася платит 35 копеек?
\end{enumerate}

\end{enumerate}
% \newpage % для выделения отдельной страницы

\section{Деревья и уравнения на ожидания}

Упражнение. Неправильную монетку с вероятностью «орла» равной $0.7$ подбрасывают до первого «орла».
Чему равно среднее количество подбрасываний?  Орлов? Решек? Какова вероятность чётного числа бросков? 

Ищем математическое ожидание. 

Через составление рекуррентного уравнения
\[
a = 0.7 \cdot 1 + 0.3 (1 + a).
\]
Через мысленное повторение большого количества экспериментов и подсчета, сколько бросков придется на одного достигнутого орла. 

Через нахождение таблички распределения и суммирования. 

Записали случайные величины количества бросков $N$ и количества решек $R$ как функции. 
Например, $N(HHT) = 3$ или $R(HHHHT)= 4$.

Вероятность для чётного бросков нашли только через суммирование (можно было уравнением).

И без формального определения ввели производящую функцию. 

Множество (событие):
\[
A = \{HT, HHHT, HHHHHT, \ldots\}
\]
Производящая функция (интересующий нас объект записанный как функция)
\[
g(H, T) = H\cdot T + H\cdot H\cdot H\cdot T + H^5T + \ldots  
\]
Вероятность
\[
\P(A) = g(0.3, 0.7) = 0.3 \cdot 0.7 + 0.3^3 0.7 + 0.3^5 0.7 + \ldots  
\]


Упражнение. Подбрасываем монетку бесконечное количество раз. 

Какова вероятность того, что последовательность HTT будет выкинута раньше THT?

Какова вероятность того, что последовательность TTH будет выкинута раньше THT?

\[
A = \{HTT \text{ выпадет раньше } THT\}, \quad  B = \{TTH \text{ выпадет раньше } THT\}
\]

Нарисовали дерево с упрощениями. Срезали «уши» и назвали этот метод «методом Ван-Гога».
На упрощенном дереве видно, что ситуация симметричная, поэтому $\P(A) = 0.5$.

Школьники в большинстве сами построили дерево для вычисления $\P(B)$. Оно уже не симметричное.
По нему составляем вместе уравнение на $b=\P(B)$:
\[
b = 0.5 + 0.25b  
\]
И получаем $b=2/3$. Замечаем чудо! Число букв одинаковое и оказывается важен их порядок!


О школьниках: было 15 человек, трое не знали, что такое геометрическая прогрессия, 
поэтому просто выводили сумму с помощью домножения и вычитания. 
Искали слагаемые с парой в двух суммой и одно слагаемое «одинокое» без пары. 

\section{Задача о ежах}

В каждой вершине треугольника по ёжику. Каждую минуту с вероятностью $0.5$ каждый ёжик
независимо от других двигается по часовой стрелке, с вероятностью
$0.5$ — против часовой стрелки.
Обозначим $T$ — время до встречи всех ёжиков в одной вершине для чаепития.

\begin{enumerate}
  \item Постройте схему возможных взаимных позиций и найдите вероятности перехода между позициями. 
  \item Найдите $\P(T=1)$, $\P(T=2)$, $\P(T=3)$, $\E(T)$.
  \item В момент каждого посещения позиции ежи получают по 100 шишек каждый. 
 Обозначим количество шишек, собранных ежами к моменту чаепития, буквой $R$. Найдите $\E(R)$.
 \item Найдите вероятность $\P(T\text{  — чётное})$.
\end{enumerate}

Удобно для наглядности обозначить вероятности перехода буквами, $\alpha$, $\beta$, $\gamma$, \ldots.
И в буквах даже $\P(T=4)$ легко выписать. 

Далее используем неожиданный трюк. Найти $\E(T)$ сразу сложно. 
Однако, легко составить систему на $\E(T \mid \text{ старт в }A)$, $\E(T \mid \text{ старт в }B)$, $\E(T \mid \text{ старт в }С)$.

Аналогичная система составляется для $\E(R)$ и для $\P(T — \text{ чётное})$.

\section{Условная вероятность}

Решили задачу про тётю Машу и двух детей и про вероятность быть больным при условии, что человек по тесту болен.

Что-то я, вероятно, не докрутил, кажется школьники не оч впечатлились. 


\section{Условная подборка}
% \newpage
\begin{enumerate}
  \item Имеется три монетки. Две «правильных» и одна — с
  «орлами» по обеим сторонам. Петя выбирает одну монетку наугад и
  подкидывает её два раза. Оба раза выпадает «орел». Какова
  условная вероятность того, что монетка «неправильная»?
\item   Два охотника одновременно выстрелили в одну утку. Первый попадает с
вероятностью 0.4, второй — с вероятностью 0.7 независимо от первого.
\begin{enumerate}
\item Какова вероятность того, что в утку попала ровно одна пуля?
\item  Какова условная вероятность того, что утка была убита первым
охотником, если в утку попала ровно одна пуля?
\end{enumerate}
\item Игрок получает 13 карт из колоды в 52 карты.
Какова вероятность, что у него как минимум два туза, если
известно, что у него есть хотя бы один туз?
Какова вероятность того, что у него как минимум два туза, если
известно, что у него есть туз пик?
\item У тети Маши — двое детей, один старше другого. Предположим, что вероятности рождения мальчика и девочки равны и не зависят от дня недели, а пол первого и второго ребенка независимы. Для каждой из четырех ситуаций найдите условную вероятность того, что у тёти Маши есть дети обоих полов.
\begin{enumerate}
\item Известно, что хотя бы один ребенок — мальчик.
\item Тетя Маша наугад выбирает одного своего
ребенка и посылает к тете Оле, вернуть учебник по теории
вероятностей. Это оказывается мальчик.
\item Известно, что старший ребенок — мальчик.
\item На вопрос: «А правда ли тетя Маша, что у вас есть сын, родившийся в пятницу?» тётя Маша ответила: «Да».
\end{enumerate}
\item У Ивана Грозного $n$ бояр. 
Каждый боярин берёт мзду независимо от других с вероятностью $1/2$.

\begin{enumerate}
  \item Какова вероятность того, что все бояре берут мзду, если случайно выбранный боярин берёт мзду?
  \item Какова вероятность того, что все бояре берут мзду, если хотя бы один из бояр берёт мзду?
\end{enumerate}
\item Есть пять закрытых дверей. За одной из них — автомобиль, за остальными — по козе.
Маша выбирает одну из дверей.
Ведущий шоу, чтобы поддержать интригу, не открывает сразу выбранную Машей дверь.
Сначала он открывает одну из дверей не выбранных Машей,
причем ради интриги ведущий не открывает сразу и дверь с автомобилем. Из возможных вариантов он выбирает равновероятно.
Допустим, ведущий открыл дверь номер 3. 
И в этот момент он предлагает Маше изменить ваш выбор двери.

Имеет ли смысл Маше изменить свой выбор?
\item Аня хватается за верёвку в форме окружности в произвольной точке.
Боря берёт мачете и с завязанными глазами разрубает
верёвку в двух случайных независимых местах. Аня забирает себе тот кусок,
за который держится. Боря забирает оставшийся кусок. Вся верёвка имеет единичную длину.
\begin{enumerate}
\item Какова вероятность того, что у Ани верёвка длиннее?
\item Какова вероятность того, что Ане досталось больше четверти веревки, если ей досталось меньше, чем Боре?
\end{enumerate}


\end{enumerate}

% \newpage

\section{todo...}

Уравнение на ожидание



Равновероятные исходы: сложные примеры

Случайные перестановки (заключенные, старушка, а-б-в, старушка два)

Статистика




\section{Загоночная работа}





\newpage

\section{Лог. КЛШ-2022}

\begin{enumerate}
  \item 
\end{enumerate}

В теховском файле \verb|\newpage| стоит, чтобы легко было скопировать секцию, для печати двух копий подряд на одном листе.
Это позволяет экономить бумагу и время при печати :)

\subsection{Плакат}





\Closesolutionfile{solution_file}

% для гиперссылок на условия
% http://tex.stackexchange.com/questions/45415
\renewenvironment{solution}[1]{%
         % add some glue
         \vskip .5cm plus 2cm minus 0.1cm%
         {\bfseries \hyperlink{problem:#1}{#1.}}%
}%
{%
}%



\section{Решения}
\input{all_solutions}


\section{Источники мудрости}

\todo[inline]{передалать потом в bib-файл}

\begin{enumerate}
\item \url{https://github.com/bdemeshev/probability_dna}
\item \url{https://github.com/bdemeshev/probability_pro}
\end{enumerate}

\printbibliography[heading=none]


\end{document}
