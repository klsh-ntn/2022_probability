% arara: xelatex: {shell: yes}
% %arara: biber
% %arara: xelatex: {shell: yes}
% %arara: xelatex: {shell: yes}

\documentclass[12pt]{article}

\usepackage{hyperref} % гиперссылки

\usepackage{tikz} % картинки в tikz
\usetikzlibrary{arrows.meta} % tikz-прибамбас для рисовки стрелочек подлиннее

\usepackage{microtype} % свешивание пунктуации

\usepackage{array} % для столбцов фиксированной ширины

\usepackage{indentfirst} % отступ в первом параграфе

\usepackage{sectsty} % для центрирования названий частей
\allsectionsfont{\centering}

\usepackage{amsmath} % куча стандартных математических плюшек
\usepackage{amssymb} % символы
\usepackage{amsthm} % теоремки

\usepackage{comment} % добавление длинных комментариев

\usepackage[top=2cm, left=1.2cm, right=1.2cm, bottom=2cm]{geometry} % размер текста на странице

\usepackage{lastpage} % чтобы узнать номер последней страницы

\usepackage{enumitem} % дополнительные плюшки для списков
%  например \begin{enumerate}[resume] позволяет продолжить нумерацию в новом списке

\usepackage{caption} % что-то делает с подписями рисунков :)

\usepackage{qcircuit} % для рисовки квантовых диаграмм
\usepackage{physics} % бракеты

\usepackage{answers} % разделение условий и ответов в упражнениях


\usepackage{fancyhdr} % весёлые колонтитулы
\pagestyle{fancy}
\lhead{Теория невероятностей}
\chead{}
\rhead{КЛШ-2022}
\lfoot{}
\cfoot{}
\rfoot{\thepage/\pageref{LastPage}}
\renewcommand{\headrulewidth}{0.4pt}
\renewcommand{\footrulewidth}{0.4pt}



\usepackage{todonotes} % для вставки в документ заметок о том, что осталось сделать
% \todo{Здесь надо коэффициенты исправить}
% \missingfigure{Здесь будет Последний день Помпеи}
% \listoftodos — печатает все поставленные \todo'шки



\usepackage{booktabs} % красивые таблицы
% заповеди из докупентации:
% 1. Не используйте вертикальные линни
% 2. Не используйте двойные линии
% 3. Единицы измерения - в шапку таблицы
% 4. Не сокращайте .1 вместо 0.1
% 5. Повторяющееся значение повторяйте, а не говорите "то же"



\usepackage{fontspec} % что-то про шрифты?
\usepackage{polyglossia} % русификация xelatex

\setmainlanguage{russian}
\setotherlanguages{english}

% download "Linux Libertine" fonts:
% http://www.linuxlibertine.org/index.php?id=91&L=1
\setmainfont{Linux Libertine O} % or Helvetica, Arial, Cambria
% why do we need \newfontfamily:
% http://tex.stackexchange.com/questions/91507/
\newfontfamily{\cyrillicfonttt}{Linux Libertine O}

\AddEnumerateCounter{\asbuk}{\russian@alph}{щ} % для списков с русскими буквами
\setlist[enumerate, 2]{label=\asbuk*),ref=\asbuk*}

%% эконометрические сокращения
\DeclareMathOperator{\Cov}{Cov}
\DeclareMathOperator{\Arg}{Arg}
\DeclareMathOperator{\Corr}{Corr}
\DeclareMathOperator{\Var}{Var}
\DeclareMathOperator{\E}{\mathbb{E}}
\newcommand \hVar{\widehat{\Var}}
\newcommand \hCorr{\widehat{\Corr}}
\newcommand \hCov{\widehat{\Cov}}
\newcommand \cN{\mathcal{N}}
\let\P\relax
\DeclareMathOperator{\P}{\mathbb{P}}

\usepackage{multicol}

\usepackage[bibencoding = auto,
backend = biber,
sorting = none,
style=alphabetic]{biblatex}

\addbibresource{forecast_everything.bib}



% делаем короче интервал в списках
\setlength{\itemsep}{0pt}
\setlength{\parskip}{0pt}
\setlength{\parsep}{0pt}




\Newassociation{sol}{solution}{solution_file}
% sol --- имя окружения внутри задач
% solution --- имя окружения внутри solution_file
% solution_file --- имя файла в который будет идти запись решений
% можно изменить далее по ходу
\Opensolutionfile{solution_file}[all_solutions]
% в квадратных скобках фактическое имя файла

% магия для автоматических гиперссылок задача-решение
\newlist{myenum}{enumerate}{3}
% \newcounter{problem}[chapter] % нумерация задач внутри глав
\newcounter{problem}[section]

\newenvironment{problem}%
{%
\refstepcounter{problem}%
%  hyperlink to solution
     \hypertarget{problem:{\thesection.\theproblem}}{} % нумерация внутри глав
     % \hypertarget{problem:{\theproblem}}{}
     \Writetofile{solution_file}{\protect\hypertarget{soln:\thesection.\theproblem}{}}
     %\Writetofile{solution_file}{\protect\hypertarget{soln:\theproblem}{}}
     \begin{myenum}[label=\bfseries\protect\hyperlink{soln:\thesection.\theproblem}{\thesection.\theproblem},ref=\thesection.\theproblem]
     % \begin{myenum}[label=\bfseries\protect\hyperlink{soln:\theproblem}{\theproblem},ref=\theproblem]
     \item%
    }%
    {%
    \end{myenum}}
% для гиперссылок обратно надо переопределять окружение
% это происходит непосредственно перед подключением файла с решениями



\theoremstyle{definition}
\newtheorem{definition}{Определение}



\begin{document}

\tableofcontents{}

\section*{Анонс}
...

\newpage
\setcounter{section}{0}

\section{Джентельменское соглашение}

Реальность и модель. 

Множество $\Omega$ — список всех исходов. 

Не знаешь как решать — рисуй дерево!

Пример дерева. Красная шапочка, все дороги ведут в усадьбу Бабушке, а встреченный волк дарит цветы, всего пять дорог.

Событие, случайная величина. 

Интуитивно: событие — утверждение об исходе эксперимента, которое может быть истинным или ложным, случайная величина — числовое описание исхода эксперимента.

Перемножение вероятностей. А почему они умножаются?

Случайные величины: время в пути, число левых поворотов, число встреченных волков. 

Табличка распределения. 

Вероятность, математическое ожидание. 

Формально, $A \subseteq \Omega$, $X : \Omega \to \mathbb R$.

Упражнение. Подбрасываем монетку до двух орлов подряд, $X$ — число бросков. 
Найдите $\P(X=2)$, $\P(X=3)$, $\P(X=4)$, $\P(X=5)$, $\P(X>5)$ и $a=\E(X)$.

Обращаем внимание: ожидаемое число левых поворотов и наиболее вероятное число левых поворотов — это разные величины.

При поиске вероятностей использовали степени букв! Это хорошая идея, чтобы потом писать производящие функции!
Например,
\[
\P(THHHHTT) = \P(TH^4T^2)  
\]

Угадайте $a$. 

Решаем через одну неизвестную:
\[
a = 0.5(a + 1) + 0.25(a + 2) + 0.25 \cdot 2  
\]

О школьниках: на первом занятии было 17 человек. 

\section{Деревья и уравнения на ожидания}

\newpage

\begin{enumerate}
  \item Роберт Адлер нажимает на кнопку «Вкл/Выкл» на пульте дистанционного
управления телевизором. Изначально телевизор включён. Батарейки
у пульта садятся, поэтому в первый раз кнопка срабатывает с вероятностью
1/2, а далее вероятность срабатывания кнопки падает, причем падает совершенное непредсказуемым образом. 

\begin{enumerate}
  \item Какова вероятность того, что после 2022 нажатий телевизор окажется включён?
  \item Кто такой Роберт Адлер?
\end{enumerate}

\item Какова вероятность того, что у здесь собравшихся есть хотя бы одно совпадение по дням рождения?
% На турнире команд ФМТ одновременно в основном составе участвует 88 школьников. 
% Какова вероятность того, что у них есть хотя бы одно совпадение по дням рождения?
А если бы нас собралось 50 человек?

\item Илья Муромец стоит на развилке у камня. От камня начинаются
ещё три дороги. Каждая из дорог оканчивается камнем.
И от каждого камня начинаются ещё три дороги. И каждые те три
дороги оканчиваются камнем\ldots И так далее до бесконечности. На
каждой дороге живёт трёхголовый Змей Горыныч. Каждый Змей
Горыныч бодрствует независимо от других с вероятностью одна третья. 

\begin{enumerate}
  \item Какова вероятность того, что ИМ встретит ЗГ, если выбирает дороги равновероятно?
  \item Какова вероятность того, что у ИМ \textit{существует} хотя бы один путь, избегающий встречи с бодрствующими ЗГ?
\end{enumerate}

\item Подбрасываем монетку бесконечное количество раз. 
\begin{enumerate}
  \item Сколько в среднем ждать до появления последовательности HTT? А до THT? 
  \item Какова вероятность того, что последовательность HTT будет выкинута раньше THT?
  \item Сколько в среднем ждать до появления HTT или THT?
\end{enumerate}

\item  Неправильную монетку с вероятностью «орла» равной $0.7$ подбрасывают до первого «орла».
Чему равно среднее количество подбрасываний?  Орлов? Решек? Какова вероятность чётного числа бросков? 

\item В каждой вершине треугольника по ёжику. Каждую минуту с вероятностью $0.5$ каждый ежик
независимо от других двигается по часовой стрелке, с вероятностью
$0.5$ — против часовой стрелки.
Обозначим $T$ — время до встречи всех ежей в одной вершине.

Найдите $\P(T=1)$, $\P(T=2)$, $\P(T=3)$, $\E(T)$.

\item Саша и Маша поженились и решили, что будут заводить новых детей до тех пор,
пока в их семье не будут дети обоих полов. Обозначим $X$ — количество детей в их семье.
Найдите $\P(X=4)$, $\E(X)$.

\item Вася подкидывает кубик\index{кубик} до тех пор, пока на кубике не выпадет единица, или пока он сам не скажет «Стоп». 
Вася получает столько рублей, сколько выпало на кубике при последнем броске. 
Вася хочет максимизировать свой ожидаемый выигрыш.
\begin{enumerate}
\item Как выглядит оптимальная стратегия? Чему равен ожидаемый выигрыш при использовании оптимальной стратегии?
\item Какова средняя продолжительность игры при использовании оптимальной стратегии?
\item Как выглядит оптимальная стратегия и чему равен ожидаемый выигрыш, если за каждое
подбрасывание Вася платит 35 копеек?
\end{enumerate}

\end{enumerate}

\newpage

\begin{enumerate}
  \item Роберт Адлер нажимает на кнопку «Вкл/Выкл» на пульте дистанционного
управления телевизором. Изначально телевизор включён. Батарейки
у пульта садятся, поэтому в первый раз кнопка срабатывает с вероятностью
1/2, а далее вероятность срабатывания кнопки падает, причем падает совершенное непредсказуемым образом. 

\begin{enumerate}
  \item Какова вероятность того, что после 2022 нажатий телевизор окажется включён?
  \item Кто такой Роберт Адлер?
\end{enumerate}

\item Какова вероятность того, что у здесь собравшихся есть хотя бы одно совпадение по дням рождения?
% На турнире команд ФМТ одновременно в основном составе участвует 88 школьников. 
% Какова вероятность того, что у них есть хотя бы одно совпадение по дням рождения?
А если бы нас собралось 50 человек?

\item Илья Муромец стоит на развилке у камня. От камня начинаются
ещё три дороги. Каждая из дорог оканчивается камнем.
И от каждого камня начинаются ещё три дороги. И каждые те три
дороги оканчиваются камнем\ldots И так далее до бесконечности. На
каждой дороге живёт трёхголовый Змей Горыныч. Каждый Змей
Горыныч бодрствует независимо от других с вероятностью одна третья. 

\begin{enumerate}
  \item Какова вероятность того, что ИМ встретит ЗГ, если выбирает дороги равновероятно?
  \item Какова вероятность того, что у ИМ \textit{существует} хотя бы один путь, избегающий встречи с бодрствующими ЗГ?
\end{enumerate}

\item Подбрасываем монетку бесконечное количество раз. 
\begin{enumerate}
  \item Сколько в среднем ждать до появления последовательности HTT? А до THT? 
  \item Какова вероятность того, что последовательность HTT будет выкинута раньше THT?
  \item Сколько в среднем ждать до появления HTT или THT?
\end{enumerate}

\item  Неправильную монетку с вероятностью «орла» равной $0.7$ подбрасывают до первого «орла».
Чему равно среднее количество подбрасываний?  Орлов? Решек? Какова вероятность чётного числа бросков? 

\item В каждой вершине треугольника по ёжику. Каждую минуту с вероятностью $0.5$ каждый ежик
независимо от других двигается по часовой стрелке, с вероятностью
$0.5$ — против часовой стрелки.
Обозначим $T$ — время до встречи всех ежей в одной вершине.

Найдите $\P(T=1)$, $\P(T=2)$, $\P(T=3)$, $\E(T)$.

\item Саша и Маша поженились и решили, что будут заводить новых детей до тех пор,
пока в их семье не будут дети обоих полов. Обозначим $X$ — количество детей в их семье.
Найдите $\P(X=4)$, $\E(X)$.

\item Вася подкидывает кубик\index{кубик} до тех пор, пока на кубике не выпадет единица, или пока он сам не скажет «Стоп». 
Вася получает столько рублей, сколько выпало на кубике при последнем броске. 
Вася хочет максимизировать свой ожидаемый выигрыш.
\begin{enumerate}
\item Как выглядит оптимальная стратегия? Чему равен ожидаемый выигрыш при использовании оптимальной стратегии?
\item Какова средняя продолжительность игры при использовании оптимальной стратегии?
\item Как выглядит оптимальная стратегия и чему равен ожидаемый выигрыш, если за каждое
подбрасывание Вася платит 35 копеек?
\end{enumerate}

\end{enumerate}
\newpage





\section{todo...}

Уравнение на ожидание

Условная вероятность

Равновероятные исходы: сложные примеры

Случайные перестановки (заключенные, старушка, а-б-в, старушка два)

Статистика




\section{Загоночная работа}





\newpage

\section{Лог. КЛШ-2022}

\begin{enumerate}
  \item 
\end{enumerate}

В теховском файле \verb|\newpage| стоит, чтобы легко было скопировать секцию, для печати двух копий подряд на одном листе.
Это позволяет экономить бумагу и время при печати :)

\subsection{Плакат}





\Closesolutionfile{solution_file}

% для гиперссылок на условия
% http://tex.stackexchange.com/questions/45415
\renewenvironment{solution}[1]{%
         % add some glue
         \vskip .5cm plus 2cm minus 0.1cm%
         {\bfseries \hyperlink{problem:#1}{#1.}}%
}%
{%
}%



\section{Решения}
\input{all_solutions}


\section{Источники мудрости}

\todo[inline]{передалать потом в bib-файл}

\begin{enumerate}
\item 
\end{enumerate}

\printbibliography[heading=none]


\end{document}
